% Chapter 2
\chapter*{\centering Kết luận chung} % Main chapter title 
\addstarredchapter{Kết luận chung}

Luận văn này trình bày một lược đồ giải số cho bài toán tối ưu dạng trong cơ học chất lỏng, dựa trên phương pháp biến phân Hadamard và phương pháp phần tử hữu hạn. Cụ thể, luận văn đã giải quyết những vấn đề sau:
\begin{itemize}
\item Xây dựng lược đồ mô phỏng số cho hệ phương trình Navier-Stokes, dựa trên phương pháp đặc trưng và phương pháp phần tử hữu hạn.
\item Đề xuất thuật toán giải số cho bài toán tối ưu dạng trong dòng chảy Stokes.
\item Mô phỏng các thí nghiệm giải số minh họa.
\end{itemize}

Lược đồ giải số được đề xuất cho bài toán tối ưu dạng trong cơ học chất lỏng có những ưu điểm sau đây:
\begin{itemize}
\item Cho phép thực hiện các biến dạng phức tạp trong quá trình tối ưu.
\item Dễ dàng mở rộng để giải quyết cho các hình dạng khác nhau, các hàm mục tiêu cũng như các mô hình cơ học khác, chẳng hạn như mô hình đàn hồi tuyến tính, dòng đối lưu tự nhiên, \ldots.
\item Chi phí của thuật toán là vừa phải.
\end{itemize}

Tính hiệu quả và độ tin cậy có lược đồ đã được thể hiện qua năm thí nghiệm giải số. Tuy nhiên, thuật toán này cũng có những nhược điểm như sau:
\begin{itemize}
\item Dạng tối ưu thu được phụ thuộc nhiều vào hình dạng ban đầu. 
\item Thuật toán chỉ làm thay đổi hình dạng chứ không làm thay đổi topology của miền.
\end{itemize} 

%----------------------------------------------------------------------------------------
\chapter*{\centering Các hướng nghiên cứu tiếp theo}
\addstarredchapter{Các hướng nghiên cứu tiếp theo}
Tác giả đề xuất các hướng nghiên cứu liên quan có thể tiếp tục phát triển từ nội dung của luận văn này:
\begin{itemize}
\item Mô phỏng các thí nghiệm giải số cho hệ phương trình Navier-Stokes cũng như cho bài toán tối ưu dạng trong không gian ba chiều.
\item Xây dựng thuật toán tìm bước giảm tối ưu.
\item Kết hợp sử dụng phương pháp tập mức ({\em level set method}) cho bài toán tối ưu dạng để thay đổi topology của miền, như được trình bày trong nghiên cứu \cite{AJT04}.
\item Mở rộng lược đồ giải số cho dòng chảy Navier-Stokes như trong \cite{DMZ08b}, cũng như xem xét các hàm mục tiêu khác, chẳng hạn hàm cực tiểu bình phương sai số hoặc cực đại tính thấm, được đề cập trong nghiên cứu \cite{ZL08}.
\item Áp dụng lược đồ hiện tại để giải quyết các bài toán ứng dụng tối ưu dạng trong nhiều mô hình vật lý khác nhau, chẳng hạn như các cấu trúc đàn hồi \cite{AJT04, Dap13}, cấu trúc truyền nhiệt \cite{YWM13}, dòng đối lưu tự nhiên \cite{AAA+14}.
\end{itemize}

Dựa trên cách tiếp cận của thuật toán đề xuất, trong quá trình thực hiện luận văn này, tác giả đã nghiên cứu xây dựng lược đồ giải số cho bài toán tối ưu dạng trong cấu trúc đàn hồi tuyến tính và đã đạt được những kết quả nhất định. Tuy nhiên, do hạn chế về thời gian thực hiện nên nghiên cứu này vẫn đang trong quá trình tiếp tục hoàn thiện.

%----------------------------------------------------------------------------------------
\chapter*{\centering Danh mục các công trình liên quan đến luận văn đã công bố}
\addstarredchapter{Danh mục các công trình liên quan đến luận văn đã công bố}
\begin{itemize}
\item[1.] L.V. Chien, N.H. Du, T.T.T. Mai, T.M. Tam. “Characteristic finite element method for natural convection problems”. May 2018. ({\em đã gửi đăng})
\item[2.] T.T.T. Mai, L.V. Chien, and P.H. Thanh. “Shape optimization for Stokes flows using sensitivity analysis and finite element method”. \textit{Applied Numerical Mathematics}. 126 (2018), pp. 160 –179. ISSN: 0168-9274.
\end{itemize}

%----------------------------------------------------------------------------------------
